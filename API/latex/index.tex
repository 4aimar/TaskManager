

\href{https://travis-ci.org/Tastyep/TaskManager}{\tt }

Task\+Manager is an asynchronous task management library using the features of C++14.

\subsubsection*{Requirements}

A compiler supporting the C++14 features.

\subsubsection*{Features}

\subparagraph*{Components}


\begin{DoxyItemize}
\item A Thread\+Pool, manages workers (threads).
\item A Task manager for running tasks asynchronously.
\item A Scheduler for scheduling tasks asynchronously.
\end{DoxyItemize}

\subparagraph*{Interface}


\begin{DoxyItemize}
\item The library exposes a module with free functions for creating managers and schedulers.
\end{DoxyItemize}

\subsubsection*{Basic Usage}

\subparagraph*{Manager}

Note\+: The following examples use chrono literals.


\begin{DoxyCode}
\{C++\}
// Create the thread pool with the initial number of threads (2 here).
Task::Module::init(2);

// Create a task manager with one worker.
auto manager = Task::Module::makeManager(1);

// Add a new task and get its future.
auto future = manager.push([] \{ return 42; \});

// Get the result from the future and print it.
std::cout << future.get() << std::endl; // Prints 42

// Not necessary here, but the stop method ensures that all launched tasks have been executed.
manager.stop().get();
\end{DoxyCode}


In the above example if we were to push more tasks, only one at a time would be executed as the manager has only one worker assigned.

\#\#\#\#\#\# Scheduler 
\begin{DoxyCode}
\{C++\}
// Create the thread pool with the initial number of threads (2 here).
Task::Module::init(2);

// Create a scheduler with one worker.
auto scheduler = Task::Module::makeScheduler(1);

// Declare the variable n.
size\_t n = 0;

// Add new tasks and get the future.
auto future = scheduler.scheduleIn("Task1", 2s, [&n] \{ n++; \});
scheduler.scheduleIn("Task2", 1s, [&n] \{ n = 41 \});

// Get the future and print the updated value.
future.get()
std::cout << n << std::endl; // Prints 42

// Not necessary here, but the stop method ensures that all the scheduled tasks have been executed.
scheduler.stop().get();
\end{DoxyCode}


The same note applies for the scheduler regarding the number of associated workers. Also an identifier (\char`\"{}\+Task\+X\char`\"{}) is provided so that periodic tasks could be removed. 